\documentclass[a4paper, 11pt, addpoints]{exam} % Adicione o parametro answers para ver as respostas

% Pacote que contém a configuracao do modelo
\usepackage{UniAval}

\begin{document}

% Define a configuracao do cabecalho
\nomeUniversidade{Universidade Tecnológica Federal do Paraná}
\logoUniversidade{fig/utfpr}
\escalaLogoUniversidade{0.06} %0.06 - UTFPR / 0.25 - UFPR
\nomeCurso{Campus Curitiba}
\nomeProfessor{Nome do Professor}
\nomeDisciplina{Nome da Disciplina}
\dataDaProva{14/05/2018}
\siglaRegistroAcademico{RA}

% Mostra o cabecalho
\info

% Mostra as questoes existentes e quanto o aluno acertou em cada uma
\pontuacao

%%%%%%%%%%%%%%%%%%%%%%%%%%%%%%%%%%%%%%%%%%%%%%%%%

%%AAG: tipos de questoes:
%  - \begin{checkboxes}
%  - \begin{choices}
%  - \begin{oneparchoices}
%  - \begin{oneparcheckboxes}
% vide exam.cls

% Documentação do exam.cls: http://www-math.mit.edu/~psh/exam/examdoc.pdf

\vspace{5pt}
    
\begin{questions}
	\question[10] 
Considerando o código abaixo, e os conceitos apresentados em aula,
apresente o teste de mesa que ilustre o comportamento do programa na memória.
Considere como valores de entrada o conjunto de $5$ elementos: $\{4, 3, 5, 2, 1\}$.

\lstinputlisting[basicstyle=\footnotesize, numbers=left, numberstyle=\tiny]{questoes/vetores/testemesa1.c}

 
	\question[5] 
Escreva as instruções (em linguagem C) para realizar as seguintes tarefas:

\begin{enumerate}[a)]
    \item Declare um vetor de inteiros de 100 elementos;
    \item Imprima na tela o valor da sétima posição do vetor;
    \item Atribua o valor $10$ na quarta posição do vetor;
    \item Multiplique os valores das posições de indice $5$ e $6$ e armazene o resultado na posição de indice $8$;
    \item Troque os valores da primeira e última posição;
    \item Some todos os valores do vetor;
\end{enumerate}
 
	\question[5] Descreva o funcionamento das funções \texttt{malloc} e \texttt{free} e apresente exemplos de sua utilização. 
	\question Tipos estruturados:
\begin{parts}
	\part[2] Descreva as vantagens da utilização de tipos estruturados.
	\part[2] Declare uma estrutura para representar uma Data. A estrutura deve conter os campos dia, mês e ano.
	\part[2] Declare uma estrutura para representar uma Atividade. A estrutura deverá conter os campos título, descrição, nota e Data de entrega. 
	\part[2] Implemente uma função que receba (por referência) uma Atividade e (por valor) a Data de hoje. 
A função deverá alterar a nota da atividade para metade se a atividade estiver atrasada 
e imprimir a mensagem ``Entregue com atraso''. 
Caso contrário imprimir a mensagem ``Entregue no prazo''. 
A função deverá obedecer a assinatura: \\\\
\texttt{void valida (Atividade * entregue, Data hoje);} \\
	\part[2] Implemente uma função que receba um vetor de Atividades e a quantidade de Atividades realizadas. 
A função deverá calcular e retornar a nota média das Atividades (considere que todas as Atividades possuem o mesmo peso).
A função deverá obedecer a assinatura:\\\\
\texttt{float media (Atividade atividades[], int quantidade);}\\
\end{parts}


	\question Listas encadeadas:
\begin{parts}
    \part[5] Descreva e compare as vantagens e desvantagens entre a utilização de vetores e listas encadeadas.
    \fullwidth{Considerando uma lista encadeada para armazenar números inteiros:}
    \part[5] Declare uma estrutura para representar um elemento da lista;
    \fullwidth{Considerando que as funções \texttt{inserir} e \texttt{remover} da lista encadeada já estão implementadas e seguem a assinatura:}
    \texttt{Item *inserir(Item *lista, int info);} \\
    \texttt{Item *remover(Item *lista, int v);}

    \part[20] Implemente uma função que receba como parâmetro uma lista encadeada.
    A função deverá obedecer a assinatura:\\\\
    \texttt{Item* separa (Item* lista);}\\\\
    Utilizando as funções \texttt{insere} e \texttt{remove} já implementadas, a função deverá:   
    \begin{subparts}
        \subpart Declarar uma nova lista vazia.
        \subpart Copiar todos os elementos pares para a nova lista
        \subpart Remover todos os elementos pares da lista original
        \subpart Retornar o endereço da nova lista
    \end{subparts}
    \label{part:separa}
    
    \part[10] Realize o teste de mesa da função implementada na letra \ref{part:separa}. Considere como valor de entrada a seguinte lista encadeada:

    \begin{tikzpicture}[every node/.style={rectangle split, rectangle split parts=2, rectangle split horizontal,minimum height=14pt}, node distance=1em, start chain,
        every join/.style={->, shorten <=-4.5pt}]
        
        \node[draw, on chain, join] { 1  };
        \node[draw, on chain, join] { 7  };
        \node[draw, on chain, join] { 5  };
        \node[draw, on chain, join] { 2  };
        \node[draw, on chain, join] { 9  };
        \node[draw, on chain, join] { 4  };
        \node[draw, on chain, join] { 6  };
        \node[draw, on chain, join] { 3  };
        \node[on chain, join] { NULL  };
        \chainlabel{chain-1.one north}{lista};
    \end{tikzpicture}  

\end{parts}
	\question Pilhas e filas:

\begin{parts}
    \part[5] Descreva as diferenças e semelhanças entre as estruturas de dados pilha e fila.
    \fullwidth{Considerando a implementação de pilha e fila utilizando vetores e que as funções de manipulação já estão implementadas e seguem a assinatura:}
    \texttt{void enfileira(int fila[], int valor, int *inicio, int* qtd);}\\
    \texttt{int desenfileira(int fila[], int *inicio, int *qtd);}\\
    \texttt{void empilha (int pilha[], int valor, int* topo);}\\
    \texttt{int desempilha (int pilha[], int* topo);}
    \part[10] Implemente uma função que receba como parâmetro uma fila. 
    A função deverá obedecer a assinatura:\\\\
    \texttt{void calcula(int fila[], int *inicio, int *qtd);}\\\\
    A função deverá:   
    \begin{subparts}
        \subpart Declarar uma pilha vazia e inicializar o topo ($-1$).
        \subpart Desenfileirar valores da fila até que a fila esteja vazia.
        \subpart Para cada valor desenfileirado da fila:
        \begin{itemize}
            \item Se o valor desenfileirado for $1$ então desenfileira mais um valor da fila e empilha na pilha.
            \item Se o valor desenfileirado for $2$ então desempilha dois valores da pilha, soma os dois valores e empilha o resultado na pilha.
            \item Se o valor desenfileirado for $3$ então desempilha um valor e imprime na tela.
        \end{itemize}
    \end{subparts}
    \label{part:calcula}
    \part[15] Realize o teste de mesa da função implementada na letra \ref{part:calcula}. Considere como valor de entrada a seguinte fila:
    
    \begin{center}
        \newcommand{\sep}{\hspace*{.8em}}
        $\fbox{1} \fbox{4} \fbox{1} \fbox{5} \fbox{2} \fbox{1} \fbox{6} \fbox{2} \fbox{3}  $\\
        {\tiny 0 \sep 1 \sep 2 \sep 3 \sep 4 \sep 5 \sep 6 \sep 7 \sep 8}
    \end{center}
    
\end{parts}


\end{questions}

%Mensagem ao final da avaliacao
% \begin{bottompar}
% 	{\bf Porque esse é meu caminho ninja!}
% 	\includegraphics[width=0.1\textwidth]{fig/aldeia.png}
% \end{bottompar}

\end{document}
